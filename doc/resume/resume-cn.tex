% Created 2013-06-08 Sat 16:19
\documentclass[11pt,dvipdfm,CJKbookmarks]{article}
\usepackage{CJK}

\usepackage[utf8]{inputenc}
\usepackage[T1]{fontenc}
\usepackage{fixltx2e}
\usepackage{graphicx}
\usepackage{longtable}
\usepackage{float}
\usepackage{wrapfig}
\usepackage{soul}
\usepackage{textcomp}
\usepackage{marvosym}
\usepackage{wasysym}
\usepackage{latexsym}
\usepackage{amssymb}
\usepackage{amstext}
\usepackage{hyperref}
\tolerance=1000
\usepackage{mycv}
\author{Bao Haojun}
\date{\today}
\title{包昊军}
\hypersetup{
  pdfkeywords={},
  pdfsubject={},
  pdfcreator={Emacs 24.3.1 (Org mode 8.0.3)}}
\begin{document}
\AtBeginDvi{\special{pdf:tounicode UTF8-UCS2}}

\begin{CJK*}{UTF8}{simsun}


\maketitle



\section{工作经历}
\label{sec-1}
\subsubsection{2011年11月 - 目前}
\label{sec-1-0-1}
\textbf{Staff Engineer}, \href{http://marvell.com}{Marvell}, 北京

\begin{itemize}
\item 在Tools团队担任架构师的职务, 负责手机工厂生产工具的设计与实现

\item 在BSP团队负责与工具相关的功能模块的设计与实现, 包括Uboot, Kernel
接口, 工厂分区等
\end{itemize}
\subsubsection{2010年3月 - 2011年10月}
\label{sec-1-0-2}

\textbf{Staff Engineer}, RayzerLink/Letou

\begin{itemize}
\item 负责基于Nvidia Tegra2芯片的平板电脑底层软件开发, 主要包括Linux
Kernel bringup, 驱动(Touch、Lcd、Sensors), Hal(硬件抽象层)的
开发等工作

\item 指导bsp新同事底层系统开发工作
\end{itemize}
\subsubsection{2008年11月 - 2010年3月}
\label{sec-1-0-3}

\textbf{Senior Engineer}, \href{http://www.borqs.com}{播思通讯}

\begin{itemize}
\item 在Tools组工作

\item 设计并实现手机开发、测试、生产以及售后等各个环节需要用到的一系列
工具
\end{itemize}

\subsubsection{2005年9月 - 2008年9月}
\label{sec-1-0-4}

\textbf{Software Engineer}, \href{http://motorola.com}{摩托罗拉},  MD/GSG

\begin{itemize}
\item 手机多媒体软件自动调试工具开发

\item 手机多媒体软件开发
\end{itemize}
\subsubsection{2004年10月 - 2005年9月}
\label{sec-1-0-5}
\textbf{Software Engineer}, 麒麟软件

\begin{itemize}
\item 企业集成应用软件测试
\end{itemize}
\section{自由软件项目}
\label{sec-2}

\subsubsection{Emacs}
\label{sec-2-0-1}

\begin{description}
\item[\href{http://github.com/baohaojun/skeleton-complete}{skeleton-complete.el}] Emacs下的补齐工具(Emacs-lisp)

\item[\href{https://github.com/baohaojun/org-jira}{org-jira.el}] Emacs下用org-mode来进行Jira开发流程管理的工具
(Emacs-lisp)
\end{description}
\subsubsection{Android}
\label{sec-2-0-2}
\begin{description}
\item[\href{https://github.com/baohaojun/BTAndroidWebViewSelection}{CrossDict}] Android下的英文字典软件, 在Google Play上发布
(Java, Android)
\end{description}
\subsubsection{Input Method}
\label{sec-2-0-3}
\begin{description}
\item[sdim] 跨所有主流平台(Win32/Linux/Mac OS甚至Emacs)的输入法
(Python, C++, ObjC, Emacs-lisp)

\item[scim-fcitx] GNU/Linux下的输入法, 基于scim和fcitx移植(C++,
GNU/Linux)
\end{description}
\subsubsection{System Software}
\label{sec-2-0-4}
\begin{description}
\item[\href{https://github.com/baohaojun/beagrep}{beagrep}] 结合搜索引擎的源代码grep工具, 0.23秒grep两G代码(C\#,
Perl)

\item[system-config] 其他一些较小的脚本/程序, 均放在 \href{https://github.com/baohaojun}{github} 上用git管理
\end{description}

\section{技术技能}
\label{sec-3}

\subsubsection{编程语言 \& 库}
\label{sec-3-0-1}
\begin{description}
\item[熟练] Python, Emacs Lisp, C, Bash, Perl, C++, Java

\item[用过] ObjC, C\#, PHP
\end{description}

\subsubsection{写作}
\label{sec-3-0-2}
\begin{description}
\item[文本] Org-mode, Emacs
\end{description}

\subsubsection{版本管理}
\label{sec-3-0-3}
Git

\subsubsection{系统管理}
\label{sec-3-0-4}
基于Debian的Linux发行版系统管理、Bash脚本编程
\section{教育}
\label{sec-4}

\subsubsection{1997 - 2001}
\label{sec-4-0-1}
本科, 控制理论与工程, 浙江大学

\subsubsection{2001 - 2004}
\label{sec-4-0-2}
硕士, 控制理论与工程, 中科院自动化所
\section{个人信息}
\label{sec-5}
\subsubsection{出生日期}
\label{sec-5-0-1}
1980年3月10日

\subsubsection{手机}
\label{sec-5-0-2}
18610314439

\subsubsection{E-mail}
\label{sec-5-0-3}
\href{mailto:baohaojun@gmail.com}{baohaojun@gmail.com}

\subsubsection{博客}
\label{sec-5-0-4}
\url{http://baohaojun.github.io}

\subsubsection{代码}
\label{sec-5-0-5}
\url{https://github.com/baohaojun}


% Emacs 24.3.1 (Org mode 8.0.3)

\end{CJK*}
\end{document}
